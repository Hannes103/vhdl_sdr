\chapter{Fazit}
Die konzeptionelle als auch technische Umsetzung des Projektes war sehr Lehrreich.

Besonders in dem Bereiche der digitalen Signalverarbeitung (Filter-Design und deren Umsetzung im \acs{FPGA}) als auch im Bereich der Empfängerarchitekturen 
konnten viele neue Erkenntnisse gewonnen werden.
Ebenfalls sehr interessant war die Erstellung des \acs{PYNQ}-Images und die damit verbundene Arbeit mit dem Linux-Kernel.

Alles in allem, erfüllt das Projekt die anfangs gestellten Erwartungen, auch wenn er tatsächliche Empfang von Funksignalen nur mäßig funktioniert.
Hauptproblem hierbei ist wohl die korrekte Auslegung des \acs{PID}-Reglers. (vergleiche: \ref{Sec:PID_opt})

Nach einigem Rumprobieren mit den Reglerparametern ist der Empfang von \acs{QPSK}-Modulierten Signalen zwar möglich, jedoch nur bei \enquote{schnellen} Symbolraten und ohne Pulse-Shaping.

\section{Verbesserungspotential}
Aufgrund der in Summe deutlich überschrittenen Projektdauer wurden an einigen Stellen Vereinfachungen getroffen die jedoch bei einer Weiterbearbeitung des Projektes
behoben werden sollte. Der folgende Abschnitt soll einige Möglichkeiten der Verbesserung des Projektes aufzeigen.

\subsection{Auslegung des PID-Reglers} \label{Sec:PID_opt}
Die Auslegung der Reglerkoeffizienten erfolgte Experimentell, in der Simulation oder auf dem \acs{FPGA} bei einem konstanten Eingangssignal.
Die so ermittelten Koeffizienten sind jedoch nicht optimal. 
Sowohl bezüglich auf die Stabilität des Regelkreises als auch in der erreichten Einschwingzeit wären durchaus noch Optimierungen denkbar. 

Eine Verbesserung hier wäre die Erstellung eines möglichst akkuraten Streckenmodelles, anhand dessen anschließend die Reglereinstellung
nach mathematischen Prinzipien erfolgen könnte, z.B. durch Polstellenplatzierung.

Schwierigkeit hier stellt die Erstellung des Streckenmodelles dar.
Zusätzliche Komplexität folgt aus der Tatsache, dass die Regelkreisauslegung im Zeit-diskreten erfolgen müsste um die vorhandenen Totzeiten des Systemes berücksichtigen zu können.

\subsection{Ergänzung der Phasen-Regelschleifen für \acs{FSK}}
Bei korrekter Einstellung der Phasen-Regelschleife wäre es möglich diese für den Empfang von \acs{FSK}-Signalen zu nutzen.

Hierbei wäre es möglich den Stellwert des Phasen-Reglers als Ausgangssignal zu nutzen um so \acs{FSK} modulierte Daten zu dekodieren.
Damit dies möglich wäre, ist es nötig diesen Stellwert dem Rechenkern zur Verfügung zu stellen. 
Dies könnte wohl am einfachsten erfolgen, indem die \acs{AXI}-Stream basierte Ausgangsschnittstelle des Reglers dupliziert und einem weiteren \acs{DMA} zugeführt wird.

\subsection{Flexibilisierung des \acs{FIR}-Filters}
Der aktuell verwendete \acs{FIR}-Kompensationsfilter bietet, im Bezug auf die Flexibilität wenig Optionen, da er nicht Re-konfigurierbar ist.
Sowohl die Dezimierungsrate als auch die Filterkoeffizienten werden aktuell während der Synthese fixiert.

Die bereits in dem IP-Core vorhandene Konfigurationsschnittstelle könnte genutzt werden um die Koeffizienten des Filters zur Laufzeit zu ändern.
Hierzu wäre es notwendig eine \acs{AXI}-Lite Registerbank zu erstellen, welche die Re-konfiguration des Filters, anhand der vom Rechenkern erhaltenen Koeffizienten, übernimmt.

Die Dezimierung könnte ebenfalls verbessert werden, indem diese im Nachgang des Filters erfolgt.
Hierbei wäre es ebenfalls möglich eine \acs{AXI}-Lite basierte Registerbank für die Konfiguration zu nutzen.

\subsection{Inbetriebnahme des zweiten \acs{ADC}-Kanals}
Im aktuellen Design wird nur ein Kanal des \acs{ADC}s verwendet.

Durch hinzufügen eines zweiten RF-Reciever IP-Blocks wäre es möglich auch den zweiten Kanal des \acs{ADC}s nutzen zu können.
Zusätzlich wäre es nötig, die notwendige \acs{AXI}-Infrastruktur anzupassen und weitere \acs{DMA}-Controller hinzuzufügen.

Die Schnittstelle für die \acs{ADC}-Konfiguration unterstützt bereits zwei Kanäle.



